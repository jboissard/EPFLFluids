%
%  untitled
%
%  Created by Johan Boissard [] on 2010-02-27.
%  Copyright (c) Johan Boissard. All rights reserved.
% hhh
\documentclass[a4paper]{amsart}


% Packages
\usepackage[french]{babel}
\usepackage[utf8]{inputenc}
\usepackage[T1]{fontenc}
\usepackage{comment}

%special math typos - see page 65 (table 196) of comprehensive latex tut
\usepackage{amsmath,amssymb,mathrsfs}
%allows to write on whole page
\usepackage{fullpage}

%package to display degree symbol
\usepackage{gensymb}

%package for page numbering and footer
\usepackage{fancyhdr}
%to get last page
\usepackage{lastpage} % \pageref{LastPage}
%allows inclusion of graphics
\usepackage{graphics}
\usepackage{ulem}
\DeclareGraphicsExtensions{.pdf, .jpeg,.jpg}
%allows drawing
%ref:http://www.texample.net/
\usepackage{tikz}

%allows inclusion of url (hyperref is better than url) 
%ref: http://www.fauskes.net/nb/latextips/
\usepackage{hyperref}

%package for chemistry ie: \ce{(NH4)2SO4 -> NH4+ + 2SO4^2-} 
%ref:www.ctan.org/tex-archive/macros/latex/contrib/mhchem/mhchem.pdf
\usepackage[version=3]{mhchem}

%comments (begin{comment})
\usepackage{comment}

% Title
\title{BioFluids}
\author{Johan Boissard}
\date\today







\pagestyle{fancyplain}
%delete current header & footer configuration
\fancyhf{} 
\renewcommand{\headrulewidth}{0pt}
%\fancyhead[EL]{Titel} %Kopfzeile links
%\fancyhead[C]{} %zentrierte Kopfzeile
%\fancyhead[LE,RO]{Name} %Kopfzeile rechts
%head separation line 
%\renewcommand{\headrulewidth}{0.4pt} 
%foot separation line
%\renewcommand{\footrulewidth}{0.4pt}
%page number
\fancyfoot[OL]{\thepage\ of \pageref{LastPage}}
%\fancyfoot[ER]{\thepage}


% Header
\setlength{\voffset}{0pt}
\setlength{\topmargin}{0in}
\setlength{\headheight}{0in}
\setlength{\headsep}{0in}

\begin{document}
%\maketitle
\twocolumn



\section{Fluid Properties}
\subsection{Fluid properties}
\subsubsection{Density}
\begin{eqnarray*}
	\rho=\frac{m}{V}&
	\begin{cases}
		\rho_\text{air}=1.23 [\text{kg/m$^3$}]\\
		\rho_\text{water}=999 [\text{kg/m$^3$}]
	\end{cases}
\end{eqnarray*}
\subsubsection{Specific Volume}
\begin{eqnarray*}
	\dot v=\frac{1}{\rho}
\end{eqnarray*}
\subsubsection{Specific weight}
\begin{eqnarray*}
	\gamma=\rho g
\end{eqnarray*}
\subsubsection{Specific gravity}
\begin{eqnarray*}
	SG=\frac{\rho}{\rho_{H_20@4\degree}}
\end{eqnarray*}
\subsection{Ideal Gas Law}
\begin{eqnarray*}
	p=\rho RT
\end{eqnarray*}
$p$ absolute pressure, $\rho$ density, $R$ gas constant (dependant of fluid), $T$ absolute Temperature.
\begin{eqnarray*}
	&\text{Absolute pressure} \\&=& \\
	&\underbrace{\text{Atmospheric pressure}}_{101 [\text{kPa}]}
	 + \text{gage pressure}
\end{eqnarray*}


\begin{comment}
	The behavior of a flowing fluid depends on various fluid properties. Viscosity, one 
	of the important properties, is responsible for the shear force produced in a moving 
	fluid. 
	Although the two fluids shown look alike (both are clear liquids and have a specific 
	gravity of 1), they behave very differently when set into motion. The very viscous 
	silicone oil is approximately 10,000 times more viscous than the water
	\subsection{No-slip condition}
	As a fluid flows near a solid surface, it "sticks" to the surface, i.e., the fluid matches the 
	velocity of the surface. This so-called "no-slip" condition is a very important one that must 
	be satisfied in any accurate analysis of fluid flow phenomena. 
	Dye injected at the bottom of a channel through which water is flowing forms a stagnant 
	layer near the bottom due to the noslip condition. As the dye filament is moved away from 
	the bottom, the motion of the water is clearly apparent. A significant velocity gradient is 
	created near the bottom.	
\end{comment}
\subsection{Fluidity}

\subsection{Viscosity}
Speed profile:
\begin{eqnarray*}
	u(y)=U\frac{y}{b}
\end{eqnarray*}
Rate of shearing strain
\begin{eqnarray*}
	\dot \gamma = \frac{U}{b}=\frac{du}{dy}
\end{eqnarray*}
Shearing stress
\begin{eqnarray*}
	\tau \propto \dot\gamma\\
	\tau=\mu\frac{du}{dy}=\frac{F}{A}\\
	T=r\times F
\end{eqnarray*}

Newtonian Fluids: $\mu$ is constant
\\
Non-Newtonian Fluids: $\mu$ is not constant and $\tau\propto\dot\gamma$ is no longer applicable.
\subsubsection{Sutherland Equation}
\begin{eqnarray*}
	\mu=\frac{CT^{3/2}}{T+S}
\end{eqnarray*}
\subsubsection{Andrade's Equation}
\begin{eqnarray*}
	\mu=De^{B/T}\\
	\ln{\mu}=\ln{D}+\frac{B}{T}
\end{eqnarray*}

\subsubsection{Kinematic viscosity}
\begin{eqnarray*}
	v=\frac{\mu}{\rho}
\end{eqnarray*}

%%%%%%% 2
\section{Fluid statics}
\textbf{Pressure is equal at every point in a fluid at rest}

\begin{eqnarray*}
	\frac{\partial p}{\partial x}=\frac{\partial p}{\partial y}=0\\
	\frac{\partial p}{\partial x}=\frac{dp}{dz}=-\gamma=-\rho g\\
	p=\gamma h+p_0
\end{eqnarray*}
\subsection{Hydrostatic Force on a Plane Surface}
\begin{eqnarray*}
	F_R=\gamma h_cA&h_c=\text{centroid}\\
	y_R=\frac{I_{xc}}{h_cA}+h_c & I_x=I_{xc}+Ah_c^2\\
	I_{xc}=\frac{1}{12}h^3w
\end{eqnarray*}

\subsection{Buoyancy}
\begin{eqnarray*}
	\mathbf{F_B}=\rho gV_{\text{immerged}}\\
	F=A\cdot P & (\text{exos})\\
	P_iV_i=P_fV_f
\end{eqnarray*}

%%%% 3
\section{Elementary Fluid Dynamics - The Bernoulli Equation}
\subsection{Bernoulli Equation}
\begin{eqnarray*}
	\frac{P}{\rho}+\frac{V^2}{2}+gz=\text{constant along a streamline}
\end{eqnarray*}
\begin{eqnarray*}
	\frac{P}{\rho}+\int\frac{V^2}{R}dn+gz=\text{constant across a streamline}
\end{eqnarray*}
\subsection{Continuity Equation}
\begin{eqnarray*}
	Q=Q_i=A_iV_i&[m^3/s]&i=1,2,\dots
\end{eqnarray*}


%%%%%%%
\section{Fluid Kinematics}
\subsection{Velocity Fields}
\begin{eqnarray*}
	\frac{dy}{dx}=\frac{v}{u}
\end{eqnarray*}
\subsection{Material Derivative}
\begin{eqnarray*}
	\frac{D( )}{Dt}=\frac{\partial ( )}{\partial t}+(\mathbf{V}\cdot\nabla)( )
\end{eqnarray*}
\subsection{Acceleration}
\begin{eqnarray*}
	\mathbf{a}
	&=&
	\begin{pmatrix}
		a_x\\
		a_y\\
		a_z\\
	\end{pmatrix}
	\nonumber\\
	&=&\frac{D \mathbf{V}}{Dt}=a_s\hat s+a_n\hat n\nonumber\\
	&=&
	\underbrace{\frac{\partial \mathbf{V}}{t}}_{\text{local}}
	+\underbrace{\mathbf{V}\cdot\nabla( \mathbf{V} )}_{\text{convective}}\nonumber\\
	&=&\frac{\partial \mathbf{V}}{t}+
	\begin{pmatrix}
		u\\
		v\\
		w
	\end{pmatrix}
	\cdot
	\begin{pmatrix}
		\frac{\partial\mathbf{V}}{\partial x} &
		\frac{\partial\mathbf{V}}{\partial y} &
		\frac{\partial\mathbf{V}}{\partial z}		
	\end{pmatrix}
	\nonumber\\
	&=&\frac{\partial \mathbf{V}}{t}+u\frac{\partial\mathbf{V}}{\partial x}+v\frac{\partial \mathbf{V}}{\partial y}+w\frac{\partial\mathbf{V}}{\partial z}\nonumber\\
	&=&
	\begin{pmatrix}
		\frac{\partial u}{\partial t} + u\frac{\partial u}{\partial x}+ v\frac{\partial u}{\partial y}+ w\frac{\partial u}{\partial z}\\
		\frac{\partial v}{\partial t} + u\frac{\partial v}{\partial x}+ v\frac{\partial v}{\partial y}+ w\frac{\partial v}{\partial z}\\
		\frac{\partial w}{\partial t} + u\frac{\partial w}{\partial x}+ v\frac{\partial w}{\partial y}+ w\frac{\partial w}{\partial z}
		
	\end{pmatrix}
\end{eqnarray*}
with $\mathbf{V}=(u,v,w)$
\subsubsection{Acceleration in streamline coordinates}
\begin{eqnarray*}
	\mathbf{a}=V\frac{\partial V}{\partial s}\mathbf{\hat s}+\frac{V^2}{R}\mathbf{\hat n}\\
	a_s=V\frac{\partial V}{\partial s}\\a_n=\frac{V^2}{R}
\end{eqnarray*}

\subsection{Reynolds Transport Theorem (RTT)}
%\includegraphics[width=.4]{img1}
\begin{eqnarray*}
	\frac{D B_{sys}}{Dt}=\frac{\partial B_{CV}}{\partial t}+\int_{CS}\rho b \mathbf{v}\cdot\mathbf{n}dA
\end{eqnarray*}

\section{Finite Control Volume Analysis}
\subsection{Continuity Equation}
In RTT, $B=m$ and $b=B/m=1$
\begin{eqnarray*}
	\frac{D M_{sys}}{Dt}&=&0\\
	%\frac{D}{Dt}\int_{sys}\rho dV&=&\frac{\partial}{\partial t}\int_{CV}\rho dV+\int_{CS}\rho \mathbf{V}\cdot \mathbf{\hat n}dA
	\frac{\partial}{\partial t}\int_{CV}\rho dV+\int_{CS}\rho \mathbf{V}\cdot \mathbf{\hat n}dA&=&0
\end{eqnarray*}

\subsection{Linear Momentum Equation}
\begin{eqnarray*}
	\frac{\partial}{\partial t}\int_{CV}\mathbf{V}\rho dV+\int_{CS}\mathbf{V}\rho \mathbf{V}\cdot \mathbf{\hat n}dA=\sum \mathbf{F_{\text{ext}}}
\end{eqnarray*}


\subsection{Energy Equation}
\begin{eqnarray*}
	\frac{\partial}{\partial t}\int_{CV}e\rho dV
	+
	\int_{CS}e\rho \mathbf{V}\cdot \mathbf{\hat n}dA
	=
	\left(\dot Q_{\text{net}}+\dot W_{\text{net}}\right)_{CV}
	\\\dot =\text{[J/s]}\dot =\text{[F$\cdot$m/s]}\dot =\text{[kg$\cdot$m/s$^2$]}
\end{eqnarray*}
\begin{eqnarray*}
	\frac{\partial}{\partial t}\int_{CV}e\rho dV
	+
	\int_{CS}\left(\check u+\frac{p}{\rho}+\frac{v^2}{2}+gz\right)\rho \mathbf{V}\cdot \mathbf{\hat n}dA
	\\
	=
	\left(\dot Q_{\text{net}}+\dot W_{\text{net}}\right)_{CV}
\end{eqnarray*}
\begin{eqnarray*}
	\dot m
	[
		\check u_{out}
		-\check u_{in}
		+
			\left(\frac{p}{\rho}\right)_{out}
		-	
			\left(\frac{p}{\rho}\right)_{in}\\
		+
			\frac{1}{2}
				\left(V_{out}^2-V_{in}^2\right)
		+
			g(z_{out}-z_{in})
	]
	\\
	=\dot Q_{net}+\dot W_{\text{net}}
\end{eqnarray*}

\begin{eqnarray*}
	\frac{P_1}{\rho}+\frac{v_1^2}{2}+gz_1
	+{w_{\text{shaft}}}
	=\frac{P_2}{\gamma}+\frac{v_2^2}{2g}+gz_2
	+\text{loss}&
	\text{ [m$^2$/s$^2$]}
	\\
	\frac{P_1}{\gamma}+\frac{v_1^2}{2g}+z_1
	+\underbrace{\frac{\dot W_{s}}{\gamma Q}}_{h_s}
	=\frac{P_2}{\gamma}+\frac{v_2^2}{2g}+z_2
	+\underbrace{\frac{\text{loss}}{g}}_{h_l}
\end{eqnarray*}
\begin{eqnarray*}
	\text{turbine head}&h_T=-(h_s+h_L)_T\\
	\text{pump head}&h_P=(h_s-h_L)_P
\end{eqnarray*}
\begin{eqnarray*}
	v=\frac{\int v^2\rho \mathbf{v}\cdot\mathbf{\hat n}dA}{\overline v A}
\end{eqnarray*}


\onecolumn
\newpage
\twocolumn
\section{Differential Analysis of Fluid Flow}
\subsection{Volumetric dilatation rate}
\begin{eqnarray*}
	\frac{1}{V}\frac{d(\delta V)}{dt}=\mathbf{\nabla}\cdot \mathbf{V}
	\\
	\mathbf{\nabla}\cdot \mathbf{V}=0\Rightarrow&\text{Incompressible fluid}
\end{eqnarray*}

\subsection{Vorticity}
\begin{eqnarray*}
	\mathbf{J}=\nabla\times \mathbf{V}&\\
	\nabla\times \mathbf{V}=0&\Rightarrow\text{Irrotational fluid}
\end{eqnarray*}
Irrotational, Bernoulli applies otherwise Euler.

\subsection{Rate rotation Vector}
\begin{eqnarray*}
	\mathbf{\omega}=\frac{1}{2} \mathbf{J}
\end{eqnarray*}

\subsection{Rate of angular deformation}
\begin{eqnarray*}
	\dot \gamma=\frac{\partial v}{\partial x} + \frac{\partial u}{\partial y}
\end{eqnarray*}

\subsection{Stream Function}
\begin{eqnarray*}
	u=\frac{\partial \psi}{\partial y}
&&	v=-\frac{\partial \psi}{\partial x}	
\\
	v_r=\frac{1}{r}\frac{\partial \psi}{\partial \theta}
&&	v_\theta=-\frac{\partial \psi}{\partial r}
\end{eqnarray*}
\subsubsection{Rate of flow}
\begin{eqnarray*}
	Q=\Psi(B) - \Psi(A) & \text{[m$^3$/s]}
\end{eqnarray*}
\subsection{Velocity Potential}
\begin{eqnarray*}
	\mathbf{v}=\nabla\phi
\end{eqnarray*}
\subsubsection{Conservation of Mass}
\begin{eqnarray*}
	\nabla\cdot \mathbf{V}=\nabla^2\phi=0
\end{eqnarray*}

\subsection{Euler's equations of Motion}
\begin{eqnarray*}
	\rho \mathbf{g}-\nabla p=\rho\left(\frac{\partial \mathbf{v}}{\partial t}+(\mathbf{v}\cdot\nabla)\mathbf{v}\right)=\rho\frac{D \mathbf{v}}{Dt}
\end{eqnarray*}

\subsection{Navier-Stokes Equations}
\subsubsection{Cartesian Coordinates}
\begin{eqnarray*}
	\rho \mathbf{g}-\nabla p+\mu\nabla^2 \mathbf{v}=\rho\left(\frac{\partial \mathbf{v}}{\partial t}+(\mathbf{v}\cdot\nabla)\mathbf{v}\right)
\end{eqnarray*}

\subsubsection{Shearing stresses}
\begin{eqnarray*}
	\begin{cases}
		\tau_{xy}=\mu\left(\frac{\partial u}{\partial y}+\frac{\partial v}{\partial x}\right)\\
		\tau_{yz}=\mu\left(\frac{\partial v}{\partial z}+\frac{\partial w}{\partial y}\right)\\
		\tau_{xz}=\mu\left(\frac{\partial w}{\partial x}+\frac{\partial u}{\partial z}\right)		
	\end{cases}
\end{eqnarray*}
%\begin{comment}

\subsubsection{Cylindrical Coordinates}
\begin{eqnarray*}
	r:
	\rho g_r
	-\frac{\partial p}{\partial r}
	\\
	\mu \Bigg[\frac{1}{r}\frac{\partial}{\partial r}\left(r \frac{\partial u_r}{\partial r}\right) 
+ \frac{1}{r^2}\frac{\partial^2 u_r}{\partial \phi^2}
	+ \frac{\partial^2 u_r}{\partial z^2}-\frac{u_r}{r^2}-\frac{2}{r^2}\frac{\partial u_\phi}{\partial \phi}\Bigg]
	\\=
	\rho\left(
	\frac{\partial u_r}{\partial t} + u_r \frac{\partial u_r}{\partial r} + \frac{u_{\phi}}{r} \frac{\partial u_r}{\partial \phi} 
	+ u_z \frac{\partial u_r}{\partial z} - \frac{u_{\phi}^2}{r}
	\right) 
	\\
	%%%
	\phi: 
	\rho g_{\phi}
	-\frac{1}{r}\frac{\partial p}{\partial \phi} 
	\\
	+ \mu 
		\left[\frac{1}{r}\frac{\partial}{\partial r}\left(r \frac{\partial u_{\phi}}{\partial r}\right) 
		+ \frac{1}{r^2}\frac{\partial^2 u_{\phi}}{\partial \phi^2} 
		+ \frac{\partial^2 u_{\phi}}{\partial z^2} 
		+ \frac{2}{r^2}\frac{\partial u_r}{\partial \phi} 
		- \frac{u_{\phi}}{r^2}\right]
	\\
	=
	\rho\left(	
		\frac{\partial u_{\phi}}{\partial t} 
		+ u_r \frac{\partial u_{\phi}}{\partial r} 
		+ \frac{u_{\phi}}{r} \frac{\partial u_{\phi}}{\partial \phi} 
		+ u_z \frac{\partial u_{\phi}}{\partial z} 
		+ \frac{u_r u_{\phi}}{r}\right) 	
	\\
	%%%
	z:
	\rho g_z
	-\frac{\partial p}{\partial z}
	\\
	+ \mu \left[\frac{1}{r}\frac{\partial}{\partial r}\left(r \frac{\partial u_z}{\partial r}\right) + \frac{1}{r^2}\frac{\partial^2 u_z}{\partial \phi^2} + \frac{\partial^2 u_z}{\partial z^2}\right]
	\\=
	\rho 
	\left(\frac{\partial u_z}{\partial t} 
	+ u_r \frac{\partial u_z}{\partial r} 
	+ \frac{u_{\phi}}{r} \frac{\partial u_z}{\partial \phi} 
	+ u_z \frac{\partial u_z}{\partial z}\right) 
\end{eqnarray*}
%\end{comment}



\subsubsection{Nabla operator in cylindrical coordinates}
\begin{eqnarray*}
	\nabla=(
	{\partial f \over \partial \rho},
	{1 \over \rho}{\partial f \over \partial \phi},
	{\partial f \over \partial z}\boldsymbol{\hat z})
	&
	r=\sqrt{\rho^2+z^2}
\end{eqnarray*}


\subsubsection{Poiseuille's Law}
\begin{eqnarray*}
	Q=\frac{\pi R^4\Delta p}{8\mu l}\\
	\overline v = \frac{R^2\Delta p}{8\mu l}&Q=\pi R^2\\
	v_{\text{max}}=-\frac{R^2}{4\mu}\frac{\partial p}{\partial z}=\frac{R^2\Delta p}{4\mu l}
	& v_{\text{max}}=2\overline v\\
	\frac{v_z}{v_{\text{max}}}=1-\left(\frac{r}{R}\right)^2
\end{eqnarray*}


\section{Dimensional Analysis}
\subsection{Buckingham Pi Theorem}
\begin{eqnarray*}
	u_1&=&f(u_2,\dots,u_k)\\
	&\rightarrow&
	\\
	\Pi_1&=&\phi(\Pi_2,\dots,\Pi_k-r)
\end{eqnarray*}
where $r$ is the number of repeating variables.
\subsection{Similitude and Models}
\begin{eqnarray*}
	\Pi_{i_{\text{model}}}=\Pi_i&\forall i
\end{eqnarray*}
$\mu\dot =FL^{-2}T, \rho=FL^{-4}T^2$


\section{Viscous Flow in Pipes}
\subsection{Reynold's Number}
\begin{eqnarray*}
	Re=\frac{\rho vD}{\mu}=\frac{vD}{\nu}<2100
\end{eqnarray*}
\subsubsection{Fully Developed Profile}
\begin{eqnarray*}
	\text{laminar} &\frac{l_e}{D}=0.06 Re
	\\
	\text{turbulent} &\frac{l_e}{D}=4.4(Re)^{1/6}
\end{eqnarray*}

\subsection{Laminar Pipe Flow (Poiseuille improved)}
\begin{eqnarray*}
	v=\frac{(\Delta p -\gamma l \sin{\theta})D^2}{32\mu l}
\end{eqnarray*}
\subsection{Major Losses (to be added to Bernoulli)}
\begin{eqnarray*}
	h_{L_{\text{Major}}}=f\frac{l}{D}\frac{v^2}{2g}
\end{eqnarray*}
\subsection{Minor Losses}
\begin{eqnarray*}
	h_{L_{\text{Minor}}}=K_L\frac{v^2}{2g} 
	&
	K_L=\frac{\Delta p}{\frac{1}{2}\rho v^2}
\end{eqnarray*}
\subsection{Moody Chart}
\begin{eqnarray*}
	f \leftrightarrow \frac{\epsilon}{D} \leftrightarrow Re
\end{eqnarray*}

\section{Boundary Layers, Drag \& Lift}
\subsection{Lift \& Drag}
\subsubsection{Drag}
\begin{eqnarray*}
	\mathscr D&=&\int dFx = \int p\cos\theta dA+\int\tau_\omega\sin\theta dA
	\\
	c_{\mathscr D}&=&\frac{\mathscr D}{\frac{1}{2}\rho U^2A}
\end{eqnarray*}
\subsubsection{Lift}
\begin{eqnarray*}
	\mathscr L&=&\int dFy = -\int p\sin\theta dA+\int\tau_\omega\cos\theta dA	
	\\
	c_{\mathscr L}&=&\frac{\mathscr L}{\frac{1}{2}\rho U^2A}
\end{eqnarray*}
\subsection{Boundary Layer for Laminar Flow}
\begin{eqnarray*}
	\delta = c\sqrt{x} \propto \sqrt{\frac{\nu}{U}}\sqrt{x}
	\\
	\tau_\omega=0.332U^{3/2}\sqrt{\frac{\rho\mu}{x}}\\
	\mathscr{D}=\int \tau_\omega dA
	\\
	\text{Power} =\frac{\text{Energy}}{t}= \mathscr{D}v
\end{eqnarray*}

\end{document}
